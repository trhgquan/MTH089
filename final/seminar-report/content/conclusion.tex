\section{Tổng kết}
Nhóm xin tổng kết lại các nội dung đã tìm hiểu trong bài báo như sau:
\begin{itemize}
\item Bài báo giới thiệu mô hình ViHealthBERT là mô hình đơn ngữ Tiếng Việt đầu tiên cho lĩnh vực y tế, sử dụng kiến trúc của BERT và bộ trọng số huấn luyện của PhoBERT. Nhóm tác giả pretrain mô hình với các task Masked Language Modeling (MLM), Next Sentence Prediction (NSP) và Capitalized Prediction (CP). Nhóm tác giả sau đó finetune mô hình với các task NER, Acronym Disambiguation và FAQ Summarization. Kết quả của ViHealthBERT tốt hơn các mô hình PhoBERT khi kiểm thử trên các tập ngữ liệu y tế Tiếng Việt. Bài báo còn giới thiệu các bộ ngữ liệu arcDrAid và FAQ Summarization là các bộ ngữ liệu Tiếng Việt về chủ đề y tế, phục vụ cho việc huấn luyện các task Acrony Disambiguation và FAQ Summarization. Ngoài ra, bài báo giới thiệu thêm một số phương pháp text mining để thu thập ngữ liệu trong quá trình xây dựng các bộ ngữ liệu chuyên ngành.

\item Tuy nhiên, nhóm nhận thấy việc pretraining task Capitalized Predicion (CP) không thể hiện được rõ tính hiệu quả như nhóm tác giả đã đặt giả thuyết ở phần đầu bài báo. Các mô hình có sử dụng pretraining task CP không đạt được hiệu quả cao nhất, thậm chí vẫn thua nhóm mô hình ViHealthBERT train với các pretraining task cơ bản (MLM và NSP). Việc mô hình ViHealthBERT đạt hiệu quả cao so với PhoBERT có thể giải thích vì ViHealthBERT dùng bộ ngữ liệu đầu vào đặc thù của ngành y tế, so với PhoBERT dùng bộ ngữ liệu cho general domain.

\item  Mặt khác, việc sử dụng syllable-level tokenizer so với word-level tokenizer có thể ảnh hưởng đến kết quả. Với một bộ ngữ liệu đã được chuẩn hóa, làm sạch các yếu tố nhiễu (e.g. URL, email, viết tắt, ..etc) thì dùng word-level tokenizer sẽ tốt hơn; với các bộ ngữ liệu thu được từ người dùng, chưa chuẩn hóa, có nhiều yếu tố nhiễu thì dùng syllable-level tokenizer sẽ cho kết quả tốt hơn. Đây là một phát hiện mới của nhóm tác giả, và có thể chứng thực được thông qua cấu trúc các bộ ngữ liệu.
\end{itemize}
Một số hướng cải tiến cho mô hình mà nhóm đề xuất bao gồm thực hiện nhiều thử nghiệm với mô hình ViHealthBERT có pretraining task CP để chứng minh độ hiệu quả của giả thuyết nhóm tác giả bài báo đưa ra. Nhóm cũng tiến hành xây dựng một ứng dụng trích xuất triệu chứng bệnh từ miêu tả của bệnh nhân, sử dụng mô hình ViHealthBERT đã finetune. Chi tiết của ứng dụng sẽ được trình bày trong Báo cáo đồ án.