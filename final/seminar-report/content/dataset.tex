\section{Ngữ liệu huấn luyện}
\subsection{Pretraining}
ViHealthBERT sử dụng các bộ ngữ liệu \textit{Text Mining Corpus} và \textit{OSCAR} cho quá trình pretraining. Vì sử dụng trọng số huấn luyện của PhoBERT nên các bộ ngữ liệu pretraining của PhoBERT cũng được đề cập trong bảng thông số các bộ ngữ liệu pretraining (bảng~\ref{tab:pretraining-stats}).

\begin{table}[H]
\centering
\begin{tabular}{|l|c|c|}
\hline
\textbf{Dataset} & \textbf{\# Sent} & \textbf{Domain} \\ \hline
Vietnamese Wikipedia & 5M & General \\ \hline
Vietnamese news & 96M & General \\ \hline
Text Mining Corpus & 4.7M & Health, Medical \\ \hline
OSCAR's selected corpus (TF) & 25M  & Health, Medical, General \\ \hline
\end{tabular}
\caption{Thông số các bộ ngữ liệu pretraining\cite{minh-EtAl:2022:LREC}}
\label{tab:pretraining-stats}
\end{table}

\subsubsection{Text Mining Corpus}
Để xây dựng bộ ngữ liệu \textit{Text Mining Corpus}, nhóm tác giả thu thập ngữ liệu từ các trang tin điện tử, các website bệnh viện, tạp chí khoa học và sách chuyên khảo về y khoa:
\begin{itemize}
\item \textbf{Các trang tin điện tử / website bệnh viện}: nhóm tác giả dùng các từ khóa liên quan đến "y học", "vắc-xin", "y sinh" hay "sức khỏe".
\item \textbf{Các tạp chí khoa học}: nhóm tác giả trích xuất phần tóm tắt (abstract) của các tạp chí Vietnamese Medical Journal, Journal of Health and Development Studies, Journal of Medicine and Pharmacy, Medical Journal of Ho Chi Minh City.
\item \textbf{Sách chuyên khảo}: nhóm tác giả dùng bản pdf sách chuyên khảo y sinh từ website \texttt{yhoctonghop.vn}, sau đó chuyển bản pdf sang phiên bản text.
\end{itemize}
Phần ngữ liệu thô được tiền xử lý (preprocess) với các bước khử nhiễu; loại bỏ email, số điện thoại, URL; loại bỏ các câu trùng lặp sử dụng thuật toán edit-distance.

\subsubsection{OSCAR's selected corpus}
Để xây dựng Bộ ngữ liệu OSCAR\cite{ortiz-suarez-etal-2020-monolingual} nhóm tác giả đã dùng các phương pháp \textbf{Term Frequency (TF)} và \textbf{Selector} vì bộ ngữ liệu OSCAR là bộ ngữ liệu cho general-domain mà ta chỉ cần trích xuất các văn bản liên quan đến chủ đề y tế cho quá trình pretraining.

\paragraph{Term Frequency (TF)}
Phương pháp Term Frequency dựa vào tần suất xuất hiện của các từ liên quan đến chủ đề y tế để xác định xem một văn bản có thuộc chủ đề y tế hay không. Các bước thực hiện bao gồm:
\begin{enumerate}
\item Dựng một từ điển với bộ ngữ liệu thu thập được.
\item Chọn 3000 từ xuất hiện trên 10000 lần liên quan tới chủ đề y tế. Thao tác có sự kiểm tra của chuyên gia để đảm bảo các từ khóa thực sự liên quan tới chủ đề y tế.
\end{enumerate}

\paragraph{Selector}
SimCSE (\textbf{Sim}ple \textbf{C}ontrastive Learning of \textbf{S}entence \textbf{E}mbeddings)\cite{simcse2021} là một phương pháp 

\subsection{Finetuning}
\subsubsection{arcDrAid}

\subsubsection{FAQ Summarization}