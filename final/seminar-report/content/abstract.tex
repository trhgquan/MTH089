\section{Tóm tắt nội dung}
Các mô hình ngôn ngữ (Language Models - LM) được xây dựng và ứng dụng rộng rãi trong lĩnh vực xử lý ngôn ngữ tự nhiên (NLP) đã đạt được nhiều kết quả đáng chú ý trong những năm gần đây. Đặc biệt, chất lượng các mô hình đơn ngữ được huấn luyện sẵn dành cho những ngôn ngữ bị hạn chế về tài nguyên ngữ liệu đã tăng đáng kể. Đối với Tiếng Việt, một số mô hình ngôn ngữ tổng quát (general-domain language models) như PhoBERT\cite{phobert} đã cho kết quả rất cao trên thực nghiệm. Mặc dù các mô hình ngôn ngữ tổng quát rất đa dạng nhưng có rất ít các mô hình ngôn ngữ đặc thù dành riêng cho một số lĩnh vực (specific-domain language models). Đã có nhiều bộ ngữ liệu Tiếng Việt dành riêng cho lĩnh vực y tế được công bố, tuy nhiên vẫn chưa có mô hình ngôn ngữ Tiếng Việt nào được phát triển và huấn luyện dành riêng cho các task liên quan tới y tế. Vì lí do đó, nhóm tác giả bài báo \textit{ViHealthBERT: Pre-trained Language Models for Vietnamese in Health Text Mining}\cite{minh-EtAl:2022:LREC} công bố bài báo với những đóng góp sau:
\begin{itemize}
\item Giới thiệu mô hình ViHealthBERT là mô hình ngôn ngữ Tiếng Việt đặc thù cho lĩnh vực y tế, đạt được state-of-the-art (SOTA) trên các downstream tasks \textit{Nhận dạng thực thể có tên (Named-Entity Recognition - NER)}, \textit{Phân định từ viết tắt (Acronym Disambiguation)} và \textit{Tóm tắt câu hỏi y khoa (FAQ Summarization)}.
\item Giới thiệu các bộ ngữ liệu \textit{Acronym Disambiguation (AD)} và \textit{Freqently Asked Question (FAQ) Summarization} là các bộ ngữ liệu đặc thù cho lĩnh vực y tế.
\end{itemize}