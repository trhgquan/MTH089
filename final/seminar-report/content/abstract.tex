\section{Tóm tắt nội dung}
Trong những năm gần đây, các mô hình ngôn ngữ (Language Models - LM) được xây dựng và ứng dụng rộng rãi trong lĩnh vực Xử lý ngôn ngữ tự nhiên (NLP), chúng cũng đạt được nhiều kết quả đáng chú ý. Đặc biệt, chất lượng các mô hình đơn ngữ được huấn luyện sẵn dành cho những ngôn ngữ ít tài nguyên ngữ liệu đã tăng đáng kể. Đáng lưu ý, mặc dù các mô hình ngôn ngữ tổng quát (general-domain language models) rất đa dạng, nhưng có rất ít các mô hình ngôn ngữ đặc thù dành riêng cho một số lĩnh vực (specific-domain language models). Vì lí do đó, nhóm tác giả bài báo \textit{ViHealthBERT: Pre-trained Language Models for Vietnamese in Health Text Mining}\cite{minh-EtAl:2022:LREC} công bố bài báo với những đóng góp sau:
\begin{itemize}
\item Giới thiệu mô hình ViHealthBERT là mô hình ngôn ngữ Tiếng Việt đặc thù cho lĩnh vực y tế, đạt được state-of-the-art (SOTA) trên các downstream tasks \textit{Nhận dạng thực thể có tên (NER)}, \textit{Phân định từ viết tắt (Acronym Disambiguation)} và \textit{Tóm tắt câu hỏi y khoa (FAQ Summarization)}.
\item Giới thiệu các bộ ngữ liệu \textit{Acronym Disambiguation (AD)} và \textit{Freqently Asked Question (FAQ) Summarization} là các bộ ngữ liệu đặc thù cho lĩnh vực y tế.
\end{itemize}