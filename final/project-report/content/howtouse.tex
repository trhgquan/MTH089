\section{Hướng dẫn sử dụng ứng dụng}
\subsection{Clone project từ GitHub}
Tiến hành clone project từ GitHub:
\begin{lstlisting}
git clone git@github.com:trhgquan/DemoViHealthBERT-NER.git
\end{lstlisting}

\subsection{Cài đặt và khởi chạy server}
\begin{enumerate}
\item Cài đặt các thư viện cần thiết (Python):
\lstset{style=mystyle}
\begin{lstlisting}[language=bash]
pip install -r requirements.txt
\end{lstlisting}

\item Cài đặt Java SDK (yêu cầu phiên bản JDK / JRE ít nhất là v1.8). Quá trình cài đặt có thể tham khảo \href{https://www.oracle.com/java/technologies/downloads/}{link này}.

\item Cài đặt VnCoreNLP cho quá trình pre-processing trước khi đưa dữ liệu vào mô hình để dự đoán.

\lstset{style=mystyle}
\begin{lstlisting}[language=bash]
bash vncorenlp.sh
\end{lstlisting}

\item Download model nhóm đã huấn luyện tại  \href{https://drive.google.com/drive/u/2/folders/19AGLo-27EeuXDkKG2JstuCgrcwB0854r}{link này}. Sau khi download model, ta giải nén model vào thư mục \texttt{model-save} sao cho cây thư mục của project có dạng như sau:
\lstinputlisting{source/dir-tree.txt}

\item Khởi chạy server
\lstset{style=mystyle}
\begin{lstlisting}[language=bash]
python main.py 
\end{lstlisting}
\end{enumerate}
Lưu ý, sau khi chạy server, ta cần ghi nhớ lại địa chỉ IP của server cho phần ứng dụng Web. Địa chỉ IP của server là địa chỉ được in ra màn hình terminal (xem hình~\ref{fig:runserver})

\subsection{Cài đặt và khởi chạy phần ứng dụng web}
\begin{enumerate}
\item Cài đặt Flutter SDK (tham khảo tại \href{https://docs.flutter.dev/get-started/install}{link}).

\item Cập nhật phiên bản mới nhất của Flutter SDK.
\lstset{style=mystyle}
\begin{lstlisting}[language=bash]
flutter channel stable
flutter upgrade
\end{lstlisting}

\item Sửa đường dẫn API thành IP của server.
\begin{itemize}
\item Mở file \texttt{web-demo-ner/lib/data/predict\_ner\_remote\_data\_source.dart}
\item Sửa địa chỉ IP ở biến \texttt{serverUrl} thành địa chỉ IP của server (đã lưu bên trên). Minh họa cho bước sửa IP ở hình~\ref{fig:update-frontend}
\end{itemize}

\item Liệt kê danh sách các device có thể sử dụng để chạy ứng dụng Flutter.
\begin{lstlisting}[language=bash]
flutter devices
\end{lstlisting}
Kết quả thực hiện câu lệnh ở hình~\ref{fig:flutter-devices}. Ở đây nhóm sử dụng \texttt{chrome} trong danh sách connected devices, ngoài ra cũng có thể sử dụng \texttt{edge} hoặc các thiết bị mobile (nếu có).

\item Chạy ứng dụng Web.
\begin{lstlisting}[language=bash]
cd web-demo-ner
flutter run -d chrome
\end{lstlisting}
\end{enumerate}

\subsection{Demo}
\href{https://youtu.be/wpURv_DAAa4}{Nhóm có thực hiện một video demo (được upload lên YouTube)}. Giao diện của ứng dụng sau khi khởi động sẽ như hình~\ref{fig:web-demo}. Với câu đầu vào \textit{Thưa bác sĩ, bệnh nhân 911 nam 27 tuổi quốc tịch Việt Nam ho đờm, chóng mặt, buồn nôn
bị tiêu chảy}, kết quả của ứng dụng đưa ra sẽ như hình~\ref{fig:web-demo-result}. 