\section{Ứng dụng mô hình ViHealthBERT xây dựng công cụ trích xuất danh sách triệu chứng từ miêu tả của bệnh nhân}
\subsection{Động lực}
Một số bệnh viện có thiết lập hệ thống tư vấn bệnh trực tuyến: người bệnh gửi thông tin cá nhân cùng miêu tả triệu chứng, bác sĩ sẽ đọc và đưa ra chẩn đoán dựa trên miêu tả của người bệnh. Tuy nhiên, đôi khi người bệnh đưa ra nhiều thông tin không liên quan, trong khi bác sĩ chỉ cần danh sách các triệu chứng cụ thể để biết được bệnh nhân đang gặp vấn đề gì, từ đó đưa ra chẩn đoán phù hợp.

Nhóm đề xuất sử dụng mô hình ViHealthBERT ứng dụng vào bài toán Medical Named Entity Recognition (Medical NER) Tiếng Việt để tự động trích xuất danh sách các triệu chứng từ miêu tả của bệnh nhân, giúp tăng hiệu suất làm việc ở các cơ sở khám chữa bệnh.

\subsection{Finetune mô hình ViHealthBERT cho tác vụ NER}
\subsubsection{Ngữ liệu huấn luyện}
Nhóm tiến hành finetune lại mô hình ViHealthBERT để giải quyết bài toán Medical NER Tiếng Việt trên tập ngữ liệu PhoNER\_COVID19\cite{truong-etal-2021-covid}. Bộ ngữ liệu có các thông số được trình bày trong bảng~\ref{tab:covid-ner-vietnamese-stats}
\begin{table}
\centering
\begin{tabular}{|l|l|l|l|l|l|}
\hline
\textbf{Entity type} & \textbf{Train} & \textbf{Valid} & \textbf{Test} & \textbf{All} \\
\hline
PATIENT\_ID & 3240 & 1276 & 2005 & 6521 \\
\hdashline
PERSON\_NAME & 349 & 188 & 318 & 855 \\
\hdashline
AGE & 682 & 361 & 582 & 1625 \\
\hdashline
GENDER & 542 & 277 & 462 & 1281 \\
\hdashline
OCCUPATION & 205 & 132 & 173 & 510 \\
\hdashline
LOCATION & 5398 & 2737 & 4441 & 12576 \\
\hdashline
ORGANIZATION & 1137 & 551 & 771 & 2459 \\
\hdashline
SYMPTOM \& DISEASE & 1439 & 766 & 1136 & 3341 \\
\hdashline
TRANSPORTATION & 226 & 87 & 193 & 506 \\
\hdashline
DATE & 2549 & 1103 & 1654 & 5306 \\
\hline
\# Entities in total & 15767 & 7478 & 11735 & 34984 \\
\hline
\# Sentences in total & 5027 & 2000 & 3000 & 10027 \\
\hline
\end{tabular}
\caption{Các thông số của tập ngữ liệu PhoNER\_COVID19\cite{truong-etal-2021-covid}}
\label{tab:covid-ner-vietnamese-stats}
\end{table}

\subsubsection{Thông số huấn luyện và kết quả}
Nhóm tác giả finetune mô hình ViHealthBERT cho tác vụ NER bằng cách sử dụng pretraining weight của ViHealthBERT với
\begin{itemize}
\item Một layer classifier đơn giản (Linear + Dropdown) hoặc
\item Một layer Conditional Random Field (CRF).
\end{itemize}
Ở đây nhóm tiến hành finetune ViHealthBERT theo cách đầu tiên (dựng một layer classifier đơn giản). Các thông số huấn luyện được mô tả trong bảng~\ref{tab:configurations}. 

Nhóm sử dụng các độ đo Precision, Recall và F1-Score để đánh giá kết quả huấn luyện. Xét bài toán phân lớp với một lớp $C$ cố định. Với \textit{TP} là số lượng mẫu thuộc về lớp $C$ được gán nhãn đúng \textit{(True Positive)}, \textit{FP} là số lượng mẫu không thuộc lớp $C$ nhưng được gán nhãn đúng \textit{(False Positive)} và \textit{FN} là số lượng mẫu thuộc lớp $C$ và được gán nhãn sai \textit{(False Negative)}. Công thức tính Precision, Recall và F1-Score khi đó được định nghĩa như sau:
\begin{equation}
\begin{aligned}
Precision~(P) &= \frac{TP}{TP + FP} \\
Recall~(R) &= \frac{TP}{TP + FN} \\
F1 &= \frac{2PR}{P + R}
\end{aligned}
\end{equation}

Mô hình sau khi huấn luyện đạt được 95\% Precision, 94\% Recall và 95\% F1-Score khi đối chiếu với tập test của bộ ngữ liệu PhoNER\_COVID-19 đã nhắc ở phần trên. Kết quả huấn luyện của mô hình được mô tả trong bảng~\ref{tab:results-test}.
\begin{table}
\centering
\begin{tabular}{|l|c|}
\hline
\# epochs & 10 \\
\hline
batch size & 16 \\
\hline
seqlen & 70 \\
\hline
learning rate & 5e-5 \\
\hline
dropout rate & 0.1 \\
\hline
\end{tabular}
\caption{Thông số finetune mô hình ViHealthBERT cho tác vụ NER}
\label{tab:configurations}
\end{table}

\begin{table}
\centering
\begin{tabular}{|l|l|l|l|}
\hline
\textbf{label}        & \textbf{precision} & \textbf{recall} & \textbf{f1-score} \\ \hline
DATE                  & 0.9909             & 0.9921          & 0.9915            \\ \hdashline
SYMPTOM\_AND\_DISEASE & 0.9322             & 0.9040          & 0.9179            \\ \hdashline
NAME                  & 0.9073             & 0.9073          & 0.9073            \\ \hdashline
LOCATION              & 0.9529             & 0.9381          & 0.9454            \\ \hdashline
PATIENT\_ID           & 0.9857             & 0.9841          & 0.9849            \\ \hdashline
TRANSPORTATION        & 0.9442             & 0.9637          & 0.9538            \\ \hdashline
GENDER                & 1.0000             & 0.9776          & 0.9887            \\ \hdashline
ORGANIZATION          & 0.9064             & 0.9182          & 0.9123            \\ \hdashline
JOB                   & 0.8659             & 0.8208          & 0.8427            \\ \hdashline
AGE                   & 0.9803             & 0.9681          & 0.9741            \\ \hline
\textbf{micro avg}    & 0.9592             & 0.9497          & 0.9545            \\ \hline
\textbf{macro avg}    & 0.9592             & 0.9497          & 0.9544            \\ \hline
\end{tabular}
\caption{Kết quả finetune mô hình ViHealthBERT cho tác vụ NER trên tập test của bộ ngữ liệu PhoNER\_COVID-19}
\label{tab:results-test}
\end{table}

\subsection{Xây dựng ứng dụng}
Ứng dụng cần nhận biết và trích xuất các triệu chứng bệnh từ đầu vào là văn bản của người dùng. Do vậy, nhóm sẽ trích xuất các từ được gán nhãn \textit{SYMPTOM\_AND\_DISEASE} trong văn bản. Một ví dụ cho input (câu văn miêu tả triệu chứng) và output (danh sách các nhãn của câu) có thể thấy trong hình~\ref{fig:example-result}.

Từ mô hình đã huấn luyện, nhóm tiến hành xây dựng ứng dụng web cho phép người dùng nhập vào một câu văn (đoạn văn). Dữ liệu được đưa vào mô hình ViHealthBERT đã được finetune để gán nhãn, sau đó bộ nhãn và câu đầu vào sẽ được đưa vào một thuật toán đơn giản để trích ra danh sách các triệu chứng. Workflow của ứng dụng có thể được tóm tắt qua hình~\ref{fig:workflow}.
\begin{figure}
\centering
\includegraphics[scale=.6]{img/workflow.png}
\caption{Workflow ứng dụng trích xuất triệu chứng}
\label{fig:workflow}
\end{figure}
Cài đặt cụ thể của ứng dụng sẽ được nêu chi tiết ở các phần sau.

\subsubsection{VnCoreNLP và word segmentation}
 Với đầu vào là đoạn văn bản \texttt{text}, nhóm dùng công cụ VnCoreNLP\footnote{\href{https://github.com/vncorenlp/VnCoreNLP}{https://github.com/vncorenlp/VnCoreNLP}} để thực hiện tác vụ tách từ (word segmentation).

\lstset{style=mystyle}
\lstinputlisting[language=Python]{source/segmentation.py}
Ví dụ:
\begin{itemize}
\item \textbf{Input}: 
\textit{Thưa bác sĩ, bệnh nhân 911 nam 27 tuổi quốc tịch Việt Nam ho đờm, chóng mặt, buồn nôn bị tiêu chảy}
\item \textbf{Output}: Thưa bác\_sĩ , bệnh\_nhân 911 nam 27 tuổi quốc\_tịch Việt\_Nam ho đờm , chóng\_mặt , buồn\_nôn bị tiêu\_chảy
\end{itemize}

\subsubsection{AutoTokenizer}
Nhóm sử dụng tokenizer của ViHealthBERT để tokenize input. Theo code mẫu của nhóm tác giả:
\begin{itemize}
    \item \texttt{all\_input\_ids}: id của từng từ trong \texttt{input\_text}
    \item \texttt{all\_attention\_mask}: attention mask của từng từ trong \texttt{input\_text}
    \item \texttt{all\_slot\_labels\_ids}: id của từng nhãn cho mỗi từ.\footnote{Ngắn gọn hơn, đây là id của các nhãn NER.}
\end{itemize} 
Nhóm cũng đặt \texttt{max\_seq\_len = 70} - chiều dài tối đa của một sentence là 70 từ.
\lstset{style=mystyle}
\lstinputlisting[language=Python]{source/tokenizer.py}

\subsubsection{Layer classifier cho tác vụ NER}
Layer classifier cho tác vụ NER bao gồm 1 layer Linear và 1 layer Dropout:
\lstinputlisting[language=Python]{source/ner-layer.py}

\subsubsection{Kiến trúc ViHealthBERT cho tác vụ NER}
Kiến trúc ViHealthBERT cho tác vụ NER bao gồm 
\begin{itemize}
\item Backbone RoBERTa - PhoBERT.
\item Layer classifier \textbf{hoặc} layer Conditional Random Field (CRF). Nếu dùng layer classifier thì hàm loss sử dụng là CrossEntropy, ngược lại dùng CRF thì hàm loss sử dụng là hàm negative log-likelihood.
\end{itemize}

\lstinputlisting[language=Python]{source/vihealthbert.py}

\subsubsection{Khởi tạo model}
Khởi tạo model với pretrained weight của ViHealthBERT\footnote{\href{https://huggingface.co/demdecuong/vihealthbert-base-word}{https://huggingface.co/demdecuong/vihealthbert-base-word}}, dùng layer classifier với \texttt{dropout rate = 0.1}.
\lstinputlisting[language=Python]{source/init-model.py}

Output từ model sẽ là \texttt{label\_id} của từ tương ứng với các \texttt{slot\_labels\_ids} là các nhãn được định nghĩa từ trước:
\begin{lstlisting}[language=Python]
labels = ["O", "B-DATE", "I-DATE", "B-NAME", "B-AGE", "B-LOCATION", "I-LOCATION", "B-JOB", "I-JOB","B-ORGANIZATION", "I-ORGANIZATION", "B-PATIENT_ID", "B-SYMPTOM_AND_DISEASE", "I-SYMPTOM_AND_DISEASE","B-GENDER", "B-TRANSPORTATION", "I-TRANSPORTATION", "I-NAME", "I-PATIENT_ID", "I-AGE", "I-GENDER"]
\end{lstlisting}
Ví dụ:
\begin{itemize}
\item \textbf{Input}: \textit{Thưa bác sĩ, bệnh nhân 911 nam 27 tuổi quốc tịch Việt Nam ho đờm, chóng mặt, buồn nôn bị tiêu chảy.}
\item \textbf{Output}:
\vietnameselst
\lstinputlisting{source/json-output.json}
\end{itemize}

\subsubsection{Hàm trích xuất triệu chứng bệnh (tương ứng với label \textbf{SYMPTOM\_AND\_DISEASE}).}
Từ danh sách các từ và label tương ứng, ta tiến hành xây dựng một thuật toán đơn giản trích xuất các triệu chứng:
\lstinputlisting[language=Python]{source/symptom-extract.py}
